\documentclass[a0,portrait]{tikzposter}

% Importar el logo
\usepackage{graphicx}
\usepackage{lipsum} % Para el texto de ejemplo, puedes eliminarlo si no lo necesitas

% Configuración de colores
\definecolor{PosterColorOne}{RGB}{108, 92, 231}  % Morado
\definecolor{PosterColorTwo}{RGB}{56, 149, 211}  % Azul
\definecolor{PosterColorThree}{RGB}{255, 255, 255}  % Blanco
\definecolor{PosterColorFour}{RGB}{254, 209, 0}  % Amarillo
\definecolor{LightYellow}{RGB}{255, 255, 204}  % Amarillo claro

% Configuración de estilos de fondo y bloque
\usetheme{Default}
\usecolorstyle[colorOne=PosterColorOne, colorTwo=PosterColorTwo, colorThree=LightYellow]{Default}

% Configuración del título y el logo
\titlegraphic{\includegraphics[height=10cm]{Logo.jpeg}}

\title{Mi primer póster}
\author{Dagoberto Bermúdez R}
\institute{Universidad Santo Tomás, Facultad de Estadística}

\begin{document}
	
	\maketitle
	
	% Bloque de Resumen
	\block{Resumen}{
		En esta sección, deberías proporcionar un breve resumen del póster. Esto incluye los objetivos principales del estudio, la metodología empleada, los resultados más importantes y las conclusiones clave. El resumen debe ser conciso y atractivo para captar el interés de los asistentes.
	}
	
	% Bloque de Introducción
	\block{Introducción}{
		En la introducción, deberías presentar el contexto del estudio. Explica por qué el tema es importante y proporciona una revisión breve de la literatura existente. También es útil plantear las preguntas de investigación o las hipótesis que tu estudio pretende abordar. Esta sección establece el escenario para el resto del póster.
	}
	
	% Bloque de Metodología
	\block{Metodología}{
		La sección de metodología debe detallar cómo llevaste a cabo tu estudio. Describe el diseño del estudio, las técnicas de muestreo, los métodos de recolección de datos y las herramientas analíticas utilizadas. Es esencial que esta sección sea clara y detallada para que otros investigadores puedan replicar tu trabajo si lo desean.
	}
	
	% Bloque de Resultados
	\block{Resultados}{
		Aquí, debes presentar los resultados principales de tu estudio. Utiliza gráficos, tablas y otros elementos visuales para hacer los resultados más comprensibles. Asegúrate de destacar los hallazgos más relevantes y cómo responden a las preguntas de investigación planteadas en la introducción.
	}
	
	% Bloque de Conclusiones
	\block{Conclusiones}{
		En las conclusiones, resume los principales hallazgos de tu estudio y discute sus implicaciones. Comenta sobre las posibles limitaciones de tu investigación y sugiere direcciones futuras para estudios adicionales. Esta sección debe proporcionar una síntesis clara de lo que se aprendió y por qué es importante.
	}
	
	% Bloque de Referencias
	\block{Referencias}{
		Lista todas las fuentes citadas en tu póster en un formato de referencia adecuado. Asegúrate de incluir todos los artículos, libros y otros recursos que mencionaste en las secciones anteriores. Usa un estilo de citación consistente, como APA o IEEE, dependiendo de la norma de tu disciplina.
	}
	
\end{document}
